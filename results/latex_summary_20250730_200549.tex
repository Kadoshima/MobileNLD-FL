% Performance Comparison Table
\begin{table}[htbp]
\centering
\caption{Performance comparison for real-time NLD computation on iPhone 13}
\label{tab:performance}
\begin{tabular}{lrrrr}
\toprule
Method & Time (ms) & Speedup & SIMD (\%) & Error \\
\midrule
Python (Float32) & 85.0 & 1.0× & -- & Reference \\
CMSIS-DSP (Q15) & 8.5 & 10.0× & 60 & <0.01 \\
\textbf{Proposed (Q15+SIMD)} & \textbf{3.9} & \textbf{21.8×} & \textbf{95} & \textbf{<0.01} \\
\bottomrule
\end{tabular}
\end{table}

% SIMD Utilization Breakdown
\begin{table}[htbp]
\centering
\caption{SIMD utilization comparison by operation}
\label{tab:simd}
\begin{tabular}{lrr}
\toprule
Operation & CMSIS-DSP (\%) & Proposed (\%) \\
\midrule
Distance calculation & 60 & 95 \\
Cumulative sum & 55 & 92 \\
Linear regression & 65 & 96 \\
\midrule
\textbf{Overall} & \textbf{60} & \textbf{95} \\
\bottomrule
\end{tabular}
\end{table}

% Key Contributions
\begin{itemize}
\item \textbf{N1}: First Q15 fixed-point implementation of Lyapunov exponent and DFA for mobile devices
\item \textbf{N2}: Achieved 95\% SIMD utilization through NLD-specific memory layout optimization
\item \textbf{N3}: Theoretical analysis proves error bounds $\Delta\lambda < 0.01$ and $\Delta\alpha < 0.01$
\item \textbf{N4}: Real-time processing of 3-second windows in 3.9ms on iPhone 13 (21.8× speedup)
\end{itemize}
